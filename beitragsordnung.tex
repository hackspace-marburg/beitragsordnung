\documentclass[parskip=half]{scrreprt} 

\usepackage[utf8]{inputenc} 
\usepackage[T1]{fontenc} 
\usepackage[ngerman]{babel} 
\usepackage{lmodern} 
\usepackage[juratotoc]{scrjura} 
\usepackage[margin=2cm, foot=1cm]{geometry}
\usepackage{eurosym}
\usepackage[pdftex]{graphicx}
\usepackage{xcolor}

\makeatletter 
\renewcommand*{\parformat}{% 
  \global\hangindent 2em 
  \makebox[2em][l]{(\thepar)\hfill}\hspace{-0,3cm} 
} 
\makeatother 

\newcommand{\TODO}[1]{\textcolor{red}{#1}}

\begin{document} 
  
\addchap{Beitragsordnung des \TODO{Hackspace Marburg e.V.} }

\begin{contract} 

\Clause{} 
Alle Vereinsmitglieder zahlen einen Mitgliedsbeitrag.

Der Mitgliedsbeitrag wird je nach Höhe montatlich oder halbjährlich erhoben.

Ehrenmitglieder sind von der Beitragszahlung befreit.


\Clause{}
Die Beitr\"age werden jeweils zum ersten Werktag des Zahlungszeitraums f\"allig bzw. eingezogen.

Dem Verein wird hierf\"ur ein SEPA-Lastschriftmandat erteilt.
Dieses ist eine Anlage zum Aufnahmeantrag.
  
\Clause{} 
Der Beitrag staffelt sich wie folgt:
\begin{itemize}
\item Natürliche Mitglieder: \TODO{2,50\euro}\ im Monat.
\item Juristische Mitglieder: \TODO{20\euro}\ im Monat.
\item Natürliche Fördermitglieder: \TODO{5\euro}\ im Monat.
\item Juristische Fördermitglieder: \TODO{40\euro}\ im Monat.
\end{itemize}

Die Beitragszahlungen sind ausdr\"ucklich Mindestbeitr\"age.

Der Mitgliedsbeitrag kann schriftlich für das nächste Abrechnungsintervall geändert werden.

Ab einem Mitgliedsbeitrag von mehr als 15 \euro\ pro Monat wird dieser monatlich erhoben, ansonsten halbjährlich.


\Clause{}
Der Verein erhebt keine Aufnahmegeb\"uhr.

\Clause{}
Au{\ss}erordentliche Ausgaben k\"onnen durch einen zus\"atzlichen Spendenaufruf gedeckt werden.

\Clause{}
Diese Beitragsordnung kann gemäß Pargraph 6, Absatz 2 der Satzung nur durch Beschluss der Mitgliederversammlung geändert werden.


\end{contract} 
  
  
 \vspace{3cm}
  
Beschlossen durch die Mitgliederversammlung am 18.11.2018.
\end{document}
