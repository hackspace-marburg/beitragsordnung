\documentclass[10pt,conference,a4paper,onecolumn,nofonttune]{IEEEtran}

\usepackage[utf8]{inputenc}
\usepackage{ngerman}
\usepackage{hyperref}
\usepackage[pdftex]{graphicx}
\usepackage{xcolor}
\usepackage{eurosym}
\usepackage{etoolbox}

\renewcommand{\thesection}{\S \arabic{section}}

\patchcmd{\section}{\centering}{}{}{}

\begin{document} 
  
\title{Beitragsordnung Hackspace Marburg e.V.}

\maketitle


\section{}
\begin{enumerate}
  \item Alle Vereinsmitglieder zahlen einen Mitgliedsbeitrag.

  \item Der Mitgliedsbeitrag wird je nach Höhe montatlich oder halbjährlich
    erhoben.

  \item Ehrenmitglieder sind von der Beitragszahlung befreit.
\end{enumerate}


\section{}
\begin{enumerate}
  \item Die Beiträge werden jeweils zum ersten Werktag des Zahlungszeitraums
    fällig bzw. eingezogen.

  \item Dem Verein wird hierfür ein SEPA-Lastschriftmandat erteilt.Dieses ist
    eine Anlage zum Aufnahmeantrag.
\end{enumerate}
  

\section{} 
\begin{enumerate}
  \item Der Beitrag staffelt sich wie folgt:
    \begin{itemize}
      \item Natürliche Mitglieder: 2,50\euro\ im Monat.
      \item Juristische Mitglieder: 20\euro\ im Monat.
      \item Natürliche Fördermitglieder: 5\euro\ im Monat.
      \item Juristische Fördermitglieder: 40\euro\ im Monat.
    \end{itemize}

  \item Die Beitragszahlungen sind ausdrücklich Mindestbeiträge.

  \item Der Mitgliedsbeitrag kann schriftlich für das nächste
    Abrechnungsintervall geändert werden.

  \item Ab einem Mitgliedsbeitrag von mehr als 15 \euro\ pro Monat wird dieser
    monatlich erhoben, ansonsten halbjährlich.
\end{enumerate}


\section{}
\begin{enumerate}
  \item Der Verein erhebt keine Aufnahmegebühr.
\end{enumerate}


\section{}
\begin{enumerate}
  \item Außerordentliche Ausgaben können durch einen zusätzlichen Spendenaufruf
    gedeckt werden.
\end{enumerate}


\section{}
\begin{enumerate}
  \item Diese Beitragsordnung kann gemäß \textit{\S 6, Absatz 2} der
    \href{https://hsmr.cc/uploads/Verein/Satzung.pdf}{Satzung}
    nur durch Beschluss der Mitgliederversammlung geändert werden.
\end{enumerate}


\section*{}
Beschlossen durch die Mitgliederversammlung am 18.11.2018.

\end{document}
